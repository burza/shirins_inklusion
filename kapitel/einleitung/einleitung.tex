%---------------------------------------------------------------------------------------------------
% Einf�hrung
%---------------------------------------------------------------------------------------------------
% \newpage
%%\part{Anfang}
\chapter{Einleitung}

\begin{flushleft}
Die Hausarbeit mit dem Thema Institutionsanalyse, wird sich mit der AWO Kindertagesst�tte Lotte Lemke auseinandersetzen. Diese �ffnete 1994 erstmalig ihre T�ren und arbeitet mit dem Situationsansatz. Zudem sind Schwerpunkte der KiTa-Arbeit: 
\end{flushleft}

\begin{itemize}
  \item Partizipation
  \item Lernwerkstattarbeit
  \item Projektarbeit 
  \item gruppen�bergreifendes Arbeiten
  \item Zuhausegruppe
  \item Sprachf�rderung
  \item t�gliche Ausfl�ge f�r Kinder ab drei Jahren
  \item Fort- und Weiterbildung der P�dagogen/innen
\end{itemize}

\begin{flushleft}
Viele der Themen in dieser Arbeit konnte ich leider nur anrei�en, da es sonst die Kapazit�t dieser Arbeit gesprengt h�tte. Trotzallem hoffe ich, dass die Hausarbeit dem  Leser einen kleinen Einblick in das KiTa Geschehen gibt.
Des Weiteren habe ich der Einfachheithalber an vielen Stellen, wenn ich von Mitarbeiterinnen und Mitarbeiter, Erzieher und Erzieherin oder P�dagogen und P�dagoginnen, nur die m�nnliche Form gew�hlt.
Da ich f�r die Erstellung der Hausarbeit mit der Papierversion des Konzeptes der KiTa gearbeitet habe - dieses aber nur in der KiTa selbst zu erhalten ist  - verweise ich auf die Internetseite der AWO Kindertagesst�tte Lotte Lemke.\footnote{Informationen zum p�dagogischen Konzept der AWO Kita Lotte Lemke, url: \url{http://www.awo-cms.de/index.php?option=com_content&view=article&id=283&Itemid=929&lang=de}, gesehen am: 25.02.2012 12:30}
\end{flushleft}


% \begin{flushleft}
% \end{flushleft}

\newpage



% Im Februar 2001 wurde bei einem Treffen von 17 Experten in Utah der Begriff \enquote{agil}\index{agil} geboren. Dieser ersetzte die bis dahin gebr�uchliche Bezeichnung \enquote{leichtgewichtige Methoden}\index{leichtgewichtige Methoden}. Dies war die Geburt agiler Softwareentwicklung und somit auch von \index{Scrum}Scrum. Vier Jahre sp�ter hat Forrester Research eine Untersuchung zur agilen Softwareentwicklung ver�ffentlicht. Dabei wurde festgestellt, dass bereits 14 \% der Unternehmen aus Nordamerika und Europa Projekte unter Zuhilfenahme von agilen Prozessen realisieren. Anfang 2010 hat Forrester Research unter dem Titel \enquote{Agile Softwareentwicklung ist Mainstream}\index{Agile Softwareentwicklung} eine weitere Studie ver�ffentlicht. Das Ergebnis der Befragung ergab, dass nun 35�\% der Teilnehmer agile Methoden in ihren Projekten einsetzen.
% Diese Zahlen beweisen, dass agile Softwareentwicklungsprozesse sich mittlerweile bei vielen Unternehmen etabliert haben und in keinem modernen Unternehmen mehr fehlen d�rfen. Agile Softwareentwicklung ist jetzt gerade einmal zehn Jahre jung, trotzdem ist es an der Zeit, den Status quo zu betrachten. Scrum ist mittlerweile eines der popul�rsten agilen Vorgehensmodelle, aus diesem Grund wird Scrum in dieser Arbeit genauer betrachtet. Dabei f�llt ein Punkt besonders ins Auge: Scrum sieht cross-funktionale Teams\index{Cross-funktionale Teams} vor, welche alle F�higkeiten besitzen, um die bevorstehende Aufgabe zu bew�ltigen. Die Rolle des Scrum-Teams und dessen Aufgabe im Projekt, sich auf Sprintziele zu einigen und Features auszuliefern, ist eines der Kernartefakte von Scrum\index{Scrum}. Des Weiteren sieht Scrum eine hohe Interaktion unter den einzelnen Teammitgliedern vor. Doch gerade dabei werden in Zusammenhang mit designspezifischen Aufgaben Konflikte in der Praxis sichtbar, die einen konstanten Prozessfluss verhindern und die es daher zu vermeiden oder zumindest zu verringern gilt.


