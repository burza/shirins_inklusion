\chapter{Die Regenbogengruppe}
  \section{Anzahl, Geschlecht und Aufenthaltzeit}
    \begin{flushleft}
      In der Regenbogengruppe, die eine Integrationsgruppe ist werden zur Zeit 16 Kinder von einer Erzieherin, einer Heilp�dagogin und einer Kinderpflegerin betreut. Von den 16 Kindern drei Kinder �ber drei mit besonderen F�rderbedarf mit juristischer Anerkennung und ein Kind unter drei Jahren mit besonderem F�rderbedarf aber gegenw�rtig noch ohne juristische Anerkennung. Von den derzeitig 16 Kindern sind acht M�dchen und acht Jungen. Die Betreungszeiten der Kindern sind sehr unterschiedlich.  F�nf Kinder haben einen Platz von 8---12 Uhr, sieben Kinder haben einen Platz von 8---14 Uhr und vier Kinder werden nach Bedarf der Eltern an einigen Tagen von 8---12 betreut und an anderen von 8---14 Uhr. Allen Eltern steht frei, nach Bedarf Betreuungstunden f�r ihre Kinder dazu zu buchen. Der Gro�teil der Regenbogen-Kinder ist mit zwei oder drei Jahren in die Kindertagesst�tte Lotte Lemke gekommen. Sieben der Kinder der Regenbogengruppe waren zuvor bereits ein Jahr in der  Nachmitttagsgruppe (12---17 Uhr oder von 14---17 Uhr) und sind im August 2011 in die Regenbogengruppe gewechselt. Ein Kind ist seit seinem ersten Lebensjahr in der Gruppe. Au�erdem sind drei der Kinder erst mit vier Jahren in die Kindertagesst�tte Lotte Lemke gekommen.
    \end{flushleft}
    
    
  \section{Altersstruktur}
    \begin{flushleft}
      Das j�ngste Kind ist ein M�dchen, sie ist dieses Jahr zwei geworden. Da sie einen besonderen F�rderbedarf hat, ist sie nicht wie �blich bei Kindern unter drei in die Familiengruppe, sondern um dem besonderen Bedarf gerecht zu werden, in die Integrationsgruppe gekommen. Hier kann durch die Anzahl der Betreungskr�fte und der Heilp�dagogin ihr eher gerecht werden. Im Alter von drei Jahren sind zwei Jungen und ein M�dchen. Des weiteren gibt es drei vierj�hrige Jungen und drei vierj�hrige M�dchen. Vier Kinder der Regenbogengruppe sind f�nf Jahre alt, davon sind drei M�dchen und zwei Junge. Au�erdem gibt es noch zwei sechsj�hrige in der Gruppe, jeweils ein Jungen und ein M�dchen.
    \end{flushleft}