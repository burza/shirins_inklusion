%---------------------------------------------------------------------------------------------------
% Zusammenfassung
%---------------------------------------------------------------------------------------------------
% \newpage
%%\part{Schluss}
\chapter{Reflexion}


\begin{flushleft}
Ans Ende dieser Hausarbeit m�chte ich mein pers�nliches Fazit �ber die Kindertagest�tte Lotte Lemke stellen.
Ich w�rde gerne mit einer Aussage von Claudia Baumann starten: 
\enquote{Du wirst ca. zwei Jahre brauchen bist du alle Facetten dieser KiTa erfasst hast} 
\end{flushleft}


\begin{flushleft}
Nach sieben Monaten als Angestellte und vier Monaten als Praktikantin diese Kindertagesst�tte, mag ich zwar nicht alle Facetten der Kindertagesst�tte erfasst haben, trotzallem habe ich mir eine Bild machen k�nnen.
Die Arbeit in der KiTa hat mich viel gelehrt, besonders seit dem ich im Zuge meines Praktikums die Heilp�dagogin begleite. Meine Sicht auf die KiTa, ihre Aufgabe und der Arbeit am Kind hat sich ver�ndert. Die P�dagogen und ins Besondere die Heilp�dagogin sind nicht nur da um die Kinder zu erziehen, sondern sie im Falle der Heilp�dagogin zu heilen, zu f�rdern und zu fordern. 
\end{flushleft}

\begin{flushleft}
Besonders gut an dem Praktikum hat mir gefallen, dass ich nicht nur als billige Arbeitskraft gesehen wurde sondern, meine L�sungen f�r Problem und meine Meinung ernst genommen wird und f�r wichtig erachtet wird. Sch�nes Beispiel sind f�r mich die Hausbesuche an denen ich teilnehmen darf. Ich bin hier nicht nur ein stiller Begleiter, sondern darf meine Beobachtungen und Erfahrungen teilen. Ich f�hle mich dadurch sehr ernstgenommen. 
Das Praktikum f�r das zweite Semester, werde ich ebenfalls in dieser Einrichtung absolvieren. Ich freue mich jetzt schon auf die neuen Aufgaben die ich gemeinsam mit meiner Anleiterin angehen werde. Wir wollen dem Team das Thema Bildungs- und Lerngeschichten vermitteln, daf�r arbeiten wir an einer gemeinsamen Ausarbeitung. Ich kann nur jedem empfehlen in dieser KiTa ein Praktikum zu absolvieren, da man in so viele Themen im Zuge der Schwerpunktarbeit einen Einblick bekommt.
\end{flushleft}